\documentclass[dvipdfmx,a4paper,landscape]{jarticle}
\usepackage[landscape,a4paper,top=7mm,bottom=7mm,left=12mm,right=12mm]{geometry}
\usepackage{graphicx}
\usepackage{booktabs}
\usepackage{color}
\usepackage{amsmath,amssymb}
\pagestyle{empty}

%% --- カラー定義 ---
\definecolor{titleblue}{rgb}{0.10,0.22,0.46}
\definecolor{secblue}{rgb}{0.16,0.36,0.64}
\definecolor{accentred}{rgb}{0.78,0.10,0.10}
\definecolor{staryellow}{rgb}{1.0,0.85,0.0}
\definecolor{lightbg}{rgb}{0.94,0.94,0.97}

%% --- セクションヘッダ ---
\newcommand{\sechead}[1]{%
  \colorbox{secblue}{\textcolor{white}{\small\textbf{\,#1\,}}}%
}

\begin{document}

%% ============================================================
%% タイトルバー
%% ============================================================
\noindent
\colorbox{titleblue}{\parbox{\dimexpr\linewidth-2\fboxsep\relax}{\centering\vspace{1.5mm}%
  \textcolor{white}{\Large\textbf{%
    DINOSAURによる鏡面反射物体の教師なしセグメンテーション}}\\[1.5mm]
  \textcolor{white}{\small%
    --- ViTバックボーンの比較と構造的限界の検証 ---}%
\vspace{1.5mm}}}

\vspace{0.5mm}
\noindent\hfill{\small 坂口 健(電気通信大学)\quad 中世 大雄(東京大学)}

\vspace{2mm}

%% ============================================================
%% Row 1: 背景と目的 | 定量結果
%% ============================================================
\noindent
\begin{minipage}[t]{0.37\linewidth}
\sechead{1.\ 背景と目的}
\vspace{2mm}

\small
\begin{itemize}
  \setlength{\itemsep}{1.5pt}
  \setlength{\parskip}{0pt}
  \setlength{\leftskip}{-4mm}
  \item \textbf{モデル}:DINOSAUR(凍結ViT+Slot Attention)
  \item \textbf{目的}:鏡面反射物体への適用可能性の検証
  \item \textbf{課題}:バックボーン選択の影響の解明
  \item \textbf{データ}:MOVi-A(金属20+混合40=60件)
\end{itemize}
\end{minipage}
\hfill
\begin{minipage}[t]{0.60\linewidth}
\sechead{2.\ 3種のバックボーン比較と定量結果}
\vspace{2mm}

\centering\small
\begin{tabular}{lccc}
\toprule
モデル & 損失種別 & FG-ARI\,($\uparrow$) & Metal FG-ARI \\
\midrule
\textbf{DINOv2 ViT-S/14} & MSE           & \textbf{0.165} & 0.185 \\
DINOv1 ViT-S/16          & ch-norm       & 0.153          & 0.154 \\
CLIP ViT-B/16            & MSE+detach    & 0.041          & 0.048 \\
\bottomrule
\end{tabular}

\vspace{2mm}
\raggedright
\quad$\Rightarrow$ \textbf{DINOv2が最高精度}(FG-ARI 0.165)\quad
$\Rightarrow$ DINOv1はスケール差で破綻 $\to$ \textbf{ch-norm損失}で解決(0.153)\quad
$\Rightarrow$ CLIPは空間弁別能力の不足により0.041
\end{minipage}

\vspace{3mm}

%% ============================================================
%% Row 2: 構造的限界の可視化(メイン貢献)
%% ============================================================
\noindent
\sechead{3.\ 構造的限界の可視化(定性分析)%
\,\textcolor{staryellow}{$\bigstar$\,本研究の中心的貢献}}

\vspace{1.5mm}

%% --- 図1: DINOv2 スロットマスク(横長ストリップ) ---
\noindent
\begin{minipage}[t]{0.72\linewidth}
\centering
\includegraphics[width=\linewidth]{figures/dinov2_result.png}\\[0.5mm]
{\scriptsize\textbf{図1}: DINOv2スロットマスク --- 16$\times$16パッチによる境界の\textbf{にじみ}}
\end{minipage}
\hfill
\begin{minipage}[t]{0.25\linewidth}
\vspace{2mm}

\small
\fbox{\parbox{\dimexpr\linewidth-2\fboxsep\relax}{%
\textcolor{accentred}{\textbf{限界1}}\\[0.5mm]
14pxパッチ起因の\\
\textbf{16$\times$16解像度の壁}\\
$\to$ CRF・SAM等が必要
}}
\end{minipage}

\vspace{2mm}

%% --- 図2: 3種比較 ---
\noindent
\begin{minipage}[t]{0.72\linewidth}
\centering
\includegraphics[width=\linewidth,height=42mm,keepaspectratio]{figures/comparison_3backbone.png}\\[0.5mm]
{\scriptsize\textbf{図2}: 3種バックボーン比較(MOVi-A金属物体)--- 金属面での\textbf{スロット混同}}
\end{minipage}
\hfill
\begin{minipage}[t]{0.25\linewidth}
\vspace{2mm}

\small
\fbox{\parbox{\dimexpr\linewidth-2\fboxsep\relax}{%
\textcolor{accentred}{\textbf{限界2}}\\[0.5mm]
BRDF Specular成分が\\
\textbf{スロット混同}を誘発\\
$\to$ 視点依存デコーダ必要
}}
\end{minipage}

\vspace{3mm}

%% ============================================================
%% Row 3: まとめ
%% ============================================================
\noindent
\sechead{4.\ まとめと今後の課題}
\vspace{1.5mm}

\noindent{\small
\textbf{(1)}\,DINOv2のバックボーンとしての優位性を確認 \hfill
\textbf{(2)}\,16$\times$16解像度の限界 $\to$ SAM等の高解像度モデルとの統合が必要 \hfill
\textbf{(3)}\,鏡面反射への対応 $\to$ 3次元反射モデルの導入が必要
}

\end{document}
